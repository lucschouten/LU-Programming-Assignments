
%
% verslag.tex   29.2.2020
% Voorbeeld LaTeX-file voor verslagen bij Kunstmatige Intelligentie
% http://www.liacs.leidenuniv.nl/~kosterswa/AI/verslag.tex
%
% Gebruik:
%   pdflatex verslag.tex
%

\documentclass[a4paper,10pt]{article}

\parindent=0pt

\usepackage{fullpage}

\frenchspacing

\usepackage{microtype}

\usepackage[english,dutch]{babel}

\usepackage{graphicx}

\usepackage{listings}
% Er zijn talloze parameters ...
\lstset{language=C++, showstringspaces=false, basicstyle=\small,
  numbers=left, numberstyle=\tiny, numberfirstline=false, breaklines=true,
  stepnumber=1, tabsize=8, 
  commentstyle=\ttfamily, identifierstyle=\ttfamily,
  stringstyle=\itshape}

\usepackage[setpagesize=false,colorlinks=true,linkcolor=red,urlcolor=blue,pdftitle={Het grote probleem},pdfauthor={Victor Erslag}]{hyperref}

\author{Victor Erslag \and Peter Robleem}
\title{Het grote probleem}

\begin{document}

\selectlanguage{dutch}

\maketitle

\section{Inleiding} 

Dit \emph{veel te korte} verslag gaat over een groot probleem.

\section{Uitleg probleem}

Tja, wat zullen we eens zeggen? Wereldvrede?

\section{Relevant werk}

De Stelling van Puk (zie~\cite{pukkie}) zegt dat 
kabouters doorgaans klein zijn. Daarom dragen ze overigens rode mutsjes.

\section{Aanpak}

We gaan als volgt te werk, zie ook Figuur~\ref{marx}.

\bigskip

\begin{figure}[!htbp]
\begin{center}
\includegraphics[height=3cm]{marxbrothers2}
\end{center}
\caption{Zeppo, Harpo, Groucho en Chico Marx [\href{http://www.marx-brothers.org}{\underline{www.marxbrothers.org}}]}\label{marx}
\end{figure}

\noindent
Zoals Groucho al zei:
``Time flies like an arrow; fruit flies like a banana''. (Sommigen vinden trouwens dat de punt voor de " moet staan.)

\section{Implementatie}

Er is gekozen voor de programmeertaal C$\stackrel{++}{}$.
Verder \ldots\ houden we het kort.

\section{Experimenten}

De resultaten van de experimenten zijn te
vinden in onderstaande tabel:

\begin{center}
\begin{tabular}{l|l|l}
experiment & tijd (sec) & uitslag\\
\hline
1 & 10 & $-7$\\
2 & 42 & 123
\end{tabular}
\end{center}
Hoe verklaren we dit? En waar is de grafiek?

\section{Conclusie}

Leuk onderzoek, veelbelovend ook. Het ging helaas 
fout als de testopstelling niet verlicht was.
In de toekomst doen we dat anders.

\begin{thebibliography}{XX}

\bibitem{pukkie}
P.~Puk, Kabouters in de Tweede Kamer,
Ons Tijdschrift 42 (2019) 12--34.

\end{thebibliography}

\section*{Appendix: Code}

Er is gebruik gemaakt van de \href{http://www.liacs.leidenuniv.nl/~kosterswa/AI/iets.cc}{\underline{skeletcode}} die te vinden is via
de website van het college.
De code van het programma is als volgt:

\smallskip

\lstinputlisting{iets.cc}

\end{document}
